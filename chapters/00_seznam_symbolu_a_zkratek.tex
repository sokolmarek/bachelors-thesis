\subsection*{Seznam symbolů}

\begin{table}[h]
	\label{tab:symboly}
	\catcode`\-=12          % Tento řádek je tam kvůli použití cline pro czech babel. Jinak to bere pomlčku jako znak a nevnímá ji jako rozsah. 
	\begin{center}
		\begin{tabular}{p{2.5cm}p{2.5cm}p{9.25cm}}
			\noalign{\hrule height 2pt}
			Symbol  & Jednotka & Význam                            \\
			\noalign{\hrule height 2pt}
			$d[n]$  & mV       & Diferencovaný signál              \\
			$f[n]$  & mV       & Původní signál                    \\
			$n$     & –        & Iterátor                          \\
			$Ww$    & s        & Délka pevného okna (Window width) \\
			$Bc[n]$ & mV       & Zpětně kumulovaný signál          \\
			$N$     & –        & Počet vzorků                      \\
			$RR$    & s        & Délka R-R intervalu               \\
			\noalign{\hrule height 2pt}
		\end{tabular}
	\end{center}
\end{table}

\clearpage 

\subsection*{Seznam zkratek}
\begin{table}[h]
	\label{tab:zkratky}
	\catcode`\-=12          % Tento řádek je tam kvůli použití cline pro czech babel. Jinak to bere pomlčku jako znak a nevnímá ji jako rozsah. 
	\begin{center}
		\begin{tabular}{p{2.5cm}p{12.25cm}}
			\noalign{\hrule height 2pt}
			Zkratka & Význam                                                                                                                                                        \\
			\noalign{\hrule height 2pt}
			EKG     & Elektrokardiogram (Electrocardiogram)                                                                                                                         \\
			AP      & Akční potenciál (Action potential)                                                                                                                            \\
			HRV     & Variabilita srdeční frekvence (Heart rate variability)                                                                                                        \\
			PSS     & Převodní systém srdeční (Cardiac conduction system)                                                                                                           \\
			SA      & Sinoatriální uzel (Sinoatrial node)                                                                                                                           \\
			AV      & Atrioventrikulární uzel (Atrioventricular node)                                                                                                               \\
			NVI     & Neuroviscerální integrace (Neurovisceral integration) 																										\\
			EMG		& Elektromyografie                                                                                                       										\\
			VLF     & Velmi nízké frekvence (Very frequency)                                                                                                                        \\
			LF      & Nízké frekvence (Low frequency)                                                                                                                               \\
			HF      & Vysoké frekvence (High frequency)                                                                                                                             \\
			VNS     & Vegetativní nervová soustava (Vegetative nervous system)                                                                                                      \\
			QRS     & Depolarizace komor (Ventricular depolarization)                                                                                                               \\
			ANS     & Autonomní nervový systém (Autonomic nervous system)                                                                                                           \\
			APG     & Akcelerovaná fotopletysmografie (Accelerated photoplethysmography)                                                                                            \\
			FFT     & Rychlá Fourierova transformace (Fast Fourier transform)                                                                                                       \\
			SGF     & Savitzky-Golay filtr                                                                                                                                          \\
			HR      & Srdeční rytmus (Heart rate)                                                                                                                                   \\
			PNS     & Parasympatický nervový systém (Parasympathetic nervous system)                                                                                                \\
			RMSSD   & Odmocnina průměru umocněných rozdílů po sobě jdoucích NN intervalů (Root mean square of the successive differences)                                           \\
			NN50    & Počet dvojic po sobě jdoucích NN intervalů lišících se o více než 50 ms (The number of pairs of successive NN (R-R) intervals that differ by more than 50 ms) \\
			pNN50   & Hodnota parametru NN50 v poměru s celkovým počtem NN intervalů (The proportion of NN50 divided by the total number of NN (R-R) intervals)                     \\
			SDNN    & Standardní odchylka všech NN intervalů (Standard deviation of the NN/RR intervals)                                                                            \\
			FIR     & Filtr s konečnou impulzní odezvou (Finite impulse response)                                                                                                                             \\
			IIR     & Filtr s nekonečnou impulzní odezvou (Infinite impulse response)                                                                                                                             \\
			DICOM   & Standardní medicínský formát (Digital Imaging and Communications in Medicine)                                                                                 \\                                                                                                                              \\
			\noalign{\hrule height 2pt}
		\end{tabular}
	\end{center}
\end{table}
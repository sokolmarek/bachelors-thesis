\subsection*{Seznam symbolů}

\begin{table}[H]
	\label{tab:symboly}
	\begin{center}
		\begin{tabular}{p{2.5cm}p{2.5cm}p{8.25cm}}
			\noalign{\hrule height 2pt}
			Symbol                      & Jednotka & Význam                                         \\
			\noalign{\hrule height 2pt}
			$A$                         & dB       & Zesílení                                       \\
			$a$                         & ms       & šířka elipsy                                   \\
			$\alpha$                    & -        & Škálovací faktor                               \\
			$b$                         & ms       & Výška elipsy                                   \\
			$Bc[n]$                     & mV       & Zpětně kumulovaný signál                       \\
			$\beta$                     & -        & Parametr Kaiserova okna                        \\
			$c_j$                       & -        & Koeficienty polynomiální regrese               \\
			$dRR$                       & -        & Normalizovaná časová série R-R intervalů       \\
			$dRRs$                      & s        & Časová série R-R intervalů                     \\
			$Fs$                        & Hz       & Vzorkovací frekvence                           \\
			$mRR$                       & -        & Normalizovaná časová série R-R intervalů       \\
			$mRRs$                      & s        & Časová série R-R intervalů                     \\
			$R$                         & ms        & Časová hodnota vybrané R vlny                 \\
			$R_m$                       & ms       & Časová hodnota doplněné R vlny                 \\
			$R_{max}$                   & mV       & Amplituda vybrané R vlny                       \\
			$RR$                        & ms       & Časová série R-R intervalů                     \\
			$\overline{RR}$             & ms       & Průměrná hodnota R-R intervalů                 \\
			$\overrightarrow{RR_{i+1}}$ & ms        & Časový vektor R-R intervalů pro osu Y          \\
			$\overrightarrow{RR_i}$     & ms        & Časový vektor R-R intervalů pro osu X          \\
			$S11$                       & -        & Hodnoty subprostoru S1 pro osu X               \\
			$S12$                       & -        & Hodnoty subprostoru S1 pro osu Y               \\
			$S21$                       & -        & Hodnoty subprostoru S2 pro osu X               \\
			$S22$                       & -        & Hodnoty subprostoru S2 pro osu Y               \\
			$SD1$                       & ms       & Směrodatná odchylka hodnot hlavní osy elipsy   \\
			$SD2$                       & ms       & Směrodatná odchylka hodnot vedlejší osy elipsy \\
			$Th_{amp}$                  & mV       & Prahová amplituda                              \\
			$Th_{RR}$                   & ms       & Prahová délka R-R intervalu                    \\
			$Th1$                       & s        & Normalizační práh pro sérii $dRRs$             \\
			$Th2$                       & s        & Normalizační práh pro sérii $mRRs$             \\
			$w[n]$                      & -        & Koeficienty Kaiserova okna                     \\
			$W_i^{L,R}$                 & ms       & Plovoucí okno                                  \\
			$X$                         & ms       & Souřadnice elipsy pro osu X po rotaci          \\
			$x$                         & ms       & Souřadnice elipsy pro osu X                    \\
			$x[n]$                      & mV       & Originální signál                              \\
			$x1$                        & ms       & Hlavní osa elipsy                              \\
			$x2$                        & ms       & Vedlejší osa elipsy                            \\
			$Y$                         & ms       & Souřadnice elipsy pro osu Y po rotaci          \\
			$y$                         & ms       & Souřadnice elipsy pro osu Y                    \\
			$y[n]$                      & mV       & Hodnoty diferencovaného signálu                \\
			$y_j$                       & mV       & Hodnoty signálu po SGF filtraci                \\
			\noalign{\hrule height 2pt}
		\end{tabular}
	\end{center}
\end{table}

\clearpage

\subsection*{Seznam zkratek}
\begin{table}[h]
	\label{tab:zkratky}
	\begin{center}
		\begin{tabular}{p{2.5cm}p{11.25cm}}
			\noalign{\hrule height 2pt}
			Zkratka & Význam                                                                                                               \\
			\noalign{\hrule height 2pt}
			AI      & Umělá inteligence (Artificial intelligence)                                                                          \\
			ANS     & Autonomní nervová soustava (Autonomic nervous system)                                                                \\
			AP      & Akční potenciál (Action potential)                                                                                   \\
			AV      & Atrioventrikulární uzel                                                                                              \\
			CSV     & Čárkou oddělené hodnoty (Comma separated values)                                                                     \\
			DP      & Dolní propust                                                                                                        \\
			EKG     & Elektrokardiogram                                                                                                    \\
			EMG     & Elektromyogram                                                                                                       \\
			FFT     & Rychlá Fourierova transformace (Fast Fourier transform)                                                              \\
			FIR     & Filtr s konečnou impulzní odezvou (Finite impulse response)                                                          \\
			GUI     & Grafické uživatelské rozhraní (Graphical User Interface)                                                             \\
			HF      & Vysoké frekvence (High frequency)                                                                                    \\
			HP      & Horní propust                                                                                                        \\
			HR      & Srdeční frekvence (Heart rate)                                                                                       \\
			HRV     & Variabilita srdeční frekvence (Heart rate variability)                                                               \\
			IIR     & Filtr s nekonečnou impulzní odezvou (Infinite impulse response)                                                      \\
			LF      & Nízké frekvence (Low frequency)                                                                                      \\
			LTI     & Lineární časově invariantní systém (Linear time-invariant system)                                                    \\
			NVI     & Neuroviscerální integrace (Neurovisceral integration)                                                                \\
			PNS     & Parasympatický nervový systém                                                                                        \\
			PP      & Pásmová propust                                                                                                      \\
			PSS     & Převodní systém srdeční                                                                                              \\
			RMSSD   & Odmocnina průměru umocněných rozdílů po sobě jdoucích N-N intervalů (Root mean square of the successive differences) \\
			SA      & Sinoatriální uzel                                                                                                    \\
			SD      & Směrodatná odchylka (Standard deviation)                                                                             \\
			SDNN    & Standardní odchylka všech N-N intervalů (Standard deviation of the N-N intervals)                                    \\
			SNS     & Sympatický nervový systém                                                                                            \\
			SW      & Software                                                                                                             \\
			SVT     & Supraventrikulární tachykardie (Supraventricular tachycardia)                                                        \\
			VLF     & Velmi nízké frekvence (Very frequency)                                                                               \\
			VNS     & Vegetativní nervová soustava (Vegetative nervous system)                                                             \\
			WLAN    & Bezdrátová lokální síť (Wireless local area network)                                                                 \\
			\noalign{\hrule height 2pt}
		\end{tabular}
	\end{center}
\end{table}
Jednou z nejčastěji sledovaných elektrofyziologických veličin je srdeční rytmus.
Je to veličina, měnící se s každým dalším úderem srdce, ovlivňována aktivitou
autonomního nervového systému. Jelikož srdeční rytmus vychází normálně ze
sinoatriálního uzlu, mluvíme o sinusovém rytmu závislém na tonu sympatiku a
parasympatiku. Proto se mezi faktory ovlivňující srdeční rytmus řadí také
kognitivní, emoční a fyzické stavy (vlivy), které jsou zároveň stěžejními
faktory pro rozbor samotného záznamu. Existuje mnoho metod, které se
specializují na různé dílčí části zpracování a analýzy EKG, určené primárně ke
sledování změn v intervalech mezi jednotlivými stahy srdce, tedy variabilitě
srdeční frekvence (HRV). Uvedené oblasti a vztahy jsou nadále popsány v
následujících kapitolách.

Přehled současného stavu se věnuje základnímu anatomickému popisu srdce společně
s úvodem do jeho fyziologie a elektrofyziologie ve spojení s neurofyziologickými
vlivy. Dále jsou zde popsány principy vyšetřovacích metod v kardiologii,
konkrétně oblast měření, zpracování a analýza elektrického záznamu srdeční
aktivity. Závěr kapitoly je věnován detailnějšímu rozboru variability srdeční
frekvence (HRV), která je předmětem analýzy zpracovaného EKG záznamu v rámci
této práce.

\subsection{Srdce}
\label{section:heart}
Srdce (cor) je orgán nepravidelného kuželovitého tvaru velkého zhruba jako pěst
dospělého člověka, skládající se převážně ze svaloviny. Tato pravidelně
oscilující pumpa zastupuje v kardiovaskulárním systému funkci krevního čerpadla
a umožňuje tak setrvalý běh organismu. Nachází se ve střední částí hrudní dutiny
(cavitas thoracica) mezi pravou a levou pleurální blánou (pleura mediastinalis
dextra et sinistra) v prostoru za sternem nazývaném mediastinum.
\cite{Memorix2017,Weinhaus2005}.

\begin{figure}[h]
	\begin{center}
		\includegraphics[width=0.7\textwidth]{../assets/anatomy/mediastinum}
		\caption{Umístění srdce v hrudní dutině mezi plícemi v mediastinu
			(Upraveno a převzato z \cite{OpenStax})}
		\label{img:mediastinum}
	\end{center}
\end{figure}

\subsubsection{Struktura srdce}
\label{section:heart_structure}
Vnitřní svalová struktura srdce je tvořena dvěma síněmi (atrium dextrum et
sinistrum) a dvěma komorami (ventriculus dexter et sinister), oddělenými
mezikomorovou přepážkou (septum interventriculare), která současně rozděluje
srdce na levé a pravé. O tok krve srdcem se starají čtyři srdeční chlopně,
fungující jako jednosměrné ventily, které jsou vidět na Obr.
\ref{img:heartanatomy}, přičemž z komory pravého srdce je přečerpávána
neokysličená krev do plic. Z komory levého srdce se pumpuje okysličená krev do
celého krevního oběhu, a proto je svalovina levé komory silnější. Rozdíl zde není
jen ve svalovině ale také v krevním tlaku, jelikož pumpování krve do tělního
oběhu vyžaduje mnohem větší tlak než do plicního oběhu. Rozlišujeme tedy malý a
velký oběh neboli krevní cirkulaci pulmonální a systémovou. Systematicky k této
cirkulaci dochází díky řetězci opakujících se elektrických a mechanických
událostí, uskutečněných během jedné časové periody, nazývané srdeční cyklus.
Konkrétně se jedná o svalovou kontrakci (systola) a svalovou relaxaci
(diastola). Blíže je tento cyklus popsán v kapitole \ref{section:cardiac_cycle}.
Průtok krve srdcem je znázorněn pomocí bílých šipek na ilustraci níže
\cite{Memorix2017,Abbasi2014,Stejfa2006}.

\begin{figure}[h]
	\begin{center}
		\includegraphics[width=1\textwidth]{../assets/anatomy/heart}
		\caption{Schéma srdce (anteriorní řez) zobrazující tok krve srdcem
			prostřednictvím bílých šipek (Upraveno a převzato z
			\cite{OpenStax})}
		\label{img:heartanatomy}
	\end{center}
\end{figure}

Vnější svalovina, obdobně jako u cév, sestává ze tří vrstev: myokard (tunica
media), endokard (tunica intima) a epikard (tunica serosa), které společně tvoří
mohutný segment srdeční stěny. Myokard (myocardium), nejširší část srdeční stěny
podléhající kontrakcím, je tvořen v závislosti na srdečním oddílu dvěma až třemi
vrstvami příčně pruhované svaloviny. Na myokard naléhá silná řídká vrstva
kolagenního vaziva a tuku neboli epikard (epicardium), ve které probíhají cévy
zásobující srdce. Poslední vrstvou, která vystýlá srdeční dutiny a mezi síněmi a
komorami tvoří mitrální chlopně, je endokrad (endocardium). Povrch srdce obaluje
vnitřní vazivový list, epikard (epicardium), který podél vnějších srdečních cév
přechází do vazivově-serózní blány, osrdečníku (perikard, pericardium). Perikard
zabraňuje krevnímu přeplnění srdce a přetíženosti srdeční svaloviny. Prostor
mezi perikardem a epikardem je vyplněn malým množstvím serózní tekutiny,
umožňující jejich vzájemný klouzavý pohyb, a nazývá se perikardiální dutina
(cavum pericardii) \cite{Memorix2017,Weinhaus2005,Dylevsky2013}.

\begin{figure}[h]
	\begin{center}
		\includegraphics[width=0.7\textwidth]{../assets/anatomy/heart_muscle}
		\caption{Perikardiální membrána a vrstvy srdeční stěny (Upraveno a
			převzato z \cite{OpenStax})}
		\label{img:heartlayers}
	\end{center}
\end{figure}

Obecně se srdeční svalovina skládá z příčně pruhované srdeční tkáně, kombinující
vlastnosti kosterního a hladkého svalstva, avšak k zajištění srdeční činnosti
obsahuje kromě svalových buněk schopných kontrakce (pracovní myokard) také
specializované kardiomyocyty (cardiomyocyti conducentes), tvořící převodní
systém srdeční (PSS, systema conducens cordis). Tento systém společně ve spojení
s autonomním nervovým systémem (ANS) tvoří pro tuto práci zásadní část, a proto
je podrobněji probrán v samostatné kapitole \cite{Memorix2017,Dylevsky2013}.

\subsubsection{Srdeční cyklus}
\label{section:cardiac_cycle}
Jak již bylo naznačeno v kapitole \ref{section:heart_structure}, dvě z hlavních
funkcí srdce jsou: přečerpat neokysličenou krev ze systémového oběhu do plic,
kde dojde k její oxygenaci a pumpovat okysličenou krev z plic zpět do všech
tkání, kde dojde znovu k její deoxygenaci a celý proces se tak znovu opakuje.
Srdeční cyklus je tedy časově sladěný průběh dvou hlavních fází začínající
systolou a končící diastolou. Systola je okamžik, kdy je ze srdce kontrakcí a za
vysokého tlaku vypuzována krev do oběhu, zatímco při diastole neboli plnící
nízkotlaké fází, se srdce krví plní. Síně i komory podléhají oběma těmto fázím a
jsou koordinovány otvíráním a zavíráním atriventrikulárních a semilunárních
chlopní. Zároveň je regulována i čerpací funkce srdce, tak aby v každém momentu
byly splněny nároky tkání na okysličenou krev
\cite{OpenStax,Weinhaus2005,Abbasi2014}.

\begin{figure}[h]
	\begin{center}
		\includegraphics[width=1\textwidth]{../assets/anatomy/cardiac_cycle}
		\caption{Schéma srdečního cyklu (srdeční revoluce) \cite{Trojan2002}}
		\label{img:cardiac_cycle}
	\end{center}
\end{figure}

\paragraph*{\textit{Plnící fáze (atriální diastola)}\\} Síně a komory jsou
relaxované. Komory srdce se pod nízkým tlakem (0--10 \si{\mmHg}) plní neokysličenou
krví přicházející ze systémového oběhu. Krev proudí do levé síně z venae cave
inferior et superior a sinus coronarius. Do lévé síně přichází krev z plicních
žil. Otevřenou trikuspidalní a mitrální chlopní proudí krev dál ze síní právě do
komor. Komory se takto naplní krví přibližně do 70--80 \% jejich kapacity
\cite{OpenStax}.

\paragraph*{\textit{Izovolumická fáze (atriální systola)}\\} Síně podléhají
kontrakci a zvyšuje se v nich tlak (80--120 \si{\mmHg}). Díky tomu je do komor
otevřenými atrioventrikulárními chlopněmi dopraveno zbylých 20--30 \% krve.
Kontrakce síní je zároveň doprovázená depolarizací, kterou na EKG reprezentuje P
vlna a trvá přibližně 100 \si{\ms}. Konec této fáze je doprovázen otevřením
semilunárních chlopní \cite{OpenStax}.

\paragraph*{\textit{Ejekční fáze (ventrikulární systola)}\\} Tlak v komorách
roste následkem jejich kontrakce, dokud není dostatečně velký, aby došlo k
otevření semilunárních chlopní. Krev je pak vypuzena z komor do tepen, přičemž v
komorách vždy zůstává konečný systolický objem (end systolic volume, ESV) krve,
který činí přibližně 50--60 \si{\ml}. Ventrikulární systola je zde doprovázena
depolarizací komor, která je na EKG záznamu reprezentovaná jako QRS komplex.
Tato fáze trvá přibližně 270 \si{\ms} \cite{OpenStax}.

\paragraph*{\textit{Fáze izovolumické relaxace (ventrikulární diastola)}\\}
Ventrikulární relaxace je doprovázená repolarizací komor a na EKG reprezentovaná
T vlnou. Trvá přibližně 430 \si{\ms}. Komory v této fázi přecházejí do stavu relaxace
a tlak v nich klesá až dojde ke zavření semilunárních chlopní. Tlak v komorách
poté klesá dále, než klesne pod hodnotu tlaku v síních. Poté začne proudit krev
ze síní do komor, což má za následek otevření trikuspidalní a mitrální chlopně.
Obě komory jsou následně v diastole \cite{OpenStax}.

\begin{figure}[h]
	\begin{center}
		\includegraphics[width=0.8\textwidth]{../assets/anatomy/cardiac_cycle_ecg}
		\caption{Vztah mezi srdečním cyklem a EKG (Upraveno a převzato z
			\cite{OpenStax})}
		\label{img:cardiac_cycle_ecg}
	\end{center}
\end{figure}


\subsubsection{Převodní systém srdeční}
\label{section:pss}
Anatomicky se tento systém skládá ze sinoatriálního uzlu (SA uzel, nodus
sinoatrialis), atrioventrikulárního uzlu (AV uzel, nodus atrioventricularis),
síňokomorového svazku (Hisův svazek, fasciculus atrioventricularis) s jeho
raménky (Tawarova raménka, crus dextrum et sinistrum fasciculi
atrioventricularis) a koncovými vlákny (Purkyňova vlákna, rami subendocardiales)
končicími ve svalovině komor. Tuto stavbu je možno vidět níže na Obr.
\ref{img:pss} \cite{Dylevsky2013}.

\begin{figure}[h]
	\begin{center}
		\includegraphics[width=0.9\textwidth]{../assets/anatomy/pss}
		\caption{Převodní systém srdeční (Upraveno a převzato z
			\cite{ecgpediaConduction})}
		\label{img:pss}
	\end{center}
\end{figure}

Z funkčního hlediska se jedná o soubor specializovaných částí myokardu. První
část tvoří buňky pracovního myokardu, jejichž hlavním úkolem je kontrakce.
Druhou část reprezentují specializované buňky převodního srdečního systému ---
kardiomyocyty --- které jsou morfologicky těžko odlišitelné, avšak funkčně, na
rozdíl od buněk pracovního myokardu mají na starosti autonomní generaci akčního
potenciálu (AP) a rychlé vedení vzruchu elektrického charakteru za účelem
podráždění myokardu (excitabilita) a vyvolání jeho stahu (systola). Ke vzniku
těchto vzruchů dochází uvnitř orgánu (automacie) a poté se šíří dále. Navzájem
jsou propojeny interkalárními disky (disci intercalares), které vznikají
spojením koncových cytoplazmatických výběžků kardiomyocytů a tvoří tak komplexní
síť, sloužící k vedení těchto vzruchů. V místech, kde toto propojení nevzniká
jsou spoje mezi jednotlivými buňkami zajištěny desmozomy a nexy umožňující
jejich vzájemnou komunikaci \cite{Dylevsky2013,Cihak2016}.

Srdce, konkrétně srdeční svalovinu (myokard) spolu s PSS, je tedy možné
charakterizovat několika hlavními funkcemi \cite{Stejfa2006}:
\begin{itemize}
	\item \textit{Automacie (chronotropie, samočinnost)} --- samočinná rytmická
	      generace elektrických impulzů pacemakerovými buňkami k podnícení
	      pravidelné kontrakce.
	\item \textit{Excitabilita (bathmotropie, dráždivost)} --- reakce na
	      podráždění elektrickým impulzem depolarizací.
	\item \textit{Konduktivita (dromotropie, vodivost)} --- vedení vzniklých
	      elektrických vzruchů celou srdeční svalovinou.
	\item \textit{Stážlivost (inotropie, kontraktilita)} --- mechanická odpověď
	      kontraktilních buněk pracovního myokardu na vzniklé elektrické
	      podněty.
\end{itemize}

V zmíněných dvou funkčních částech je také potřeba rozlišovat rozdíly na buněčné
úrovní, a to konkrétně v rámci membránových potenciálů.

Buňky pracovního myokardu sice nemají schopnost samovolně vytvářet vzruchy ale
jejich dráždění vyvolává šíření akčního potenciálu napříč myokardem, což má za
následek zahájení kontrakce. Jejich klidový membránový potenciál se pohybuje
okolo -90 mV. Změnou klidového membránového potenciálu vzniká akční potenciál,
jehož časový průběh sestává z následujících fázi: rychlá depolarizace (fáze 0),
časná repolarizace (fáze 1), plató akčního potenciálu (fáze 2), konečná
repolarizace (fáze 3) a návrat ke klidovému potenciálu (fáze 4)
\cite{Petrek2019}.

\begin{figure}[h]
	\centering
	\setkeys{Gin}{width=\linewidth}
	\begin{subfigure}{0.4\textwidth}
		\includegraphics[width=1\textwidth]{../assets/figures/myokard_ap}
		\caption{Akční potenciál buňky pracovního myokardu \cite{Petrek2019}}
		\label{fig:myokard_ap}
	\end{subfigure}
	\hfil
	\begin{subfigure}{0.5\textwidth}
		\includegraphics[width=1\textwidth]{../assets/figures/pss_ap}
		\caption{Akční potenciál buňky převodního systému srdce
			\cite{Petrek2019}}
		\label{fig:pss_ap}
	\end{subfigure}
	\caption{Rozdíl průběhů akčních potenciálu pracovního myokardu a převodního
		systému srdce: 0 --- rychlá depolarizace, 1 --- časná repolarizace, 2
		--- plató akčního potenciálu, 3 --- konečná repolarizace, 4 --- návrat
		ke klidovému potenciálu}
	\label{fig:ap}
\end{figure}

Klidový membránový potenciál pacemakerových buněk --- buněk SA a AV uzlu --- je
nižší, než u buněk pracovního myokardu (-50 až -70 mV) a samovolně klesá k
prahové hodnotě (spontánní diastolická depolarizace). Jakmile klidový potenciál
dosáhne prahové hodnoty vzniká další akční potenciál. Průběh akčního potenciálu
se od buněk pracovního myokardu také liší. Chybí zde 1. a 2. fáze. Přehledněji
je možné pozorovat rozdíly porovnáním křivek AP na Obr. \ref{fig:ap}
\cite{Petrek2019}.

Elektrické vzruchy, vyvolávající rytmické smršťování srdečního svalu, primárně
vznikají v SA uzlu, uloženém při ústí horní duté žíly, ve stěně pravé síně.
Tento uzel je udavatel srdečního rytmu (HR) neboli primární pacemaker a vzruchy se z
něj šíří dále systémem. Než jsou tyto impulzy převedeny přes Hisův svazek na
Purkyňova vlákna, prochází vzruch AV uzlem (sekundární pacemaker), kde dochází k
jeho zpomalení a tvorbě časové prodlevy, což má za následek postupné kontrakce
síní a komor neboli posloupnost systol a diastol. Spojení mezi SA uzlem a AV
uzlem je realizováno internodálními síňovými spoji, což jsou vlákna stejného
charakteru jako u PSS. Tyto spoje umožňují rozvádět vzruchy z SA uzlu rychleji
než samotná svalovina síní a excitovat tak AV uzlík v krátkých intervalech.
Vzruchy je pak možno mezí síněmi a komorami vést pomalu či rychleji, což
představuje jeden z regulačních mechanismů srdeční frekvence
\cite{Dylevsky2013,Cihak2016}.

Tento sled, pravidelnost srdečního rytmu a proměnlivost srdečních akcí vůči
změnám v organismu, zajišťuje několikastupňový regulační systém. Zachycením
těchto elektrických proudů v čase spolu se změnami jejich potenciálu vzniká tzv.
elektrokardiogram (EKG), na kterém je možné pozorovat průběh elektrického
srdečního cyklu s jeho jednotlivými fázemi (Obr. \ref{img:pss} --- P, Q, R, S,
T, U) \cite{Dylevsky2013,Cihak2016}.


\subsubsection{Regulace srdeční frekvence}
\label{section:hr_regulation}
Změny v srdeční frekvenci (chronotropie) a její variabilitě jsou jedním z
následků regulace srdeční činnosti, která mimo jiné ovliňuje také inotropii,
dromotropii a bathmotropii. Regulaci dělíme na dvě úrovně v závislosti na místě
průběhu regulačního děje, a to na intrakardiální a extrakardiální.
Intrakardiální regulační děje probíhají v srdci samotném, například následkem
mechanických změn myokardu. Blíže tyto děje popisuje například Starlingův zákon
nebo Bainbridgeův reflex \cite{Kittnar2020}. Extrakardiální děje se dělí dále na
nervové a humorální. Jelikož je srdeční frekvence řízena hlavně nervově, tak se
tato kapitole věnuje podrobněji extrakardiálním vlivům \cite{Orel2019}.

\paragraph*{\textit{Nervová regulace srdečního rytmu}\\} Srdce je inervováno
autonomním (vegetativním) nervovým systémem, konkrétně pregangliovými
parasympatickými vlákny (rami cardiaci nervi vagi) z bloudivého nervu (nervus
vagus) a sympatickými vlákny z krčního kmene sympatiku (n. cardiacus cervicalis
superior, medius et inferior), společně s větvemi jeho horního hrudního úseku
(nn. cardiaci thoracici). Tato vlákna realizují regulační děje, vzniklé na
úrovní prodloužené míchy (medulla oblongata) v kardioexcitačním nebo
kardioinhibičním centru. Inervace tohoto typu představují jeden z vyšších stupňů
regulačních mechanismů a jejich dráždění má vliv na srdeční frekvenci a její
variabilitu \cite{Dylevsky2013,Petrek2019,Kittnar2020}.

\begin{figure}[h]
	\begin{center}
		\includegraphics[width=1\textwidth]{../assets/anatomy/hr_regulation}
		\caption{Autonomní invervace kardioexcitačních a kardioinhibičních
			oblastí nacházejících se v prodloužené míše a jejich vliv na
			normální sinusový rytmus(Upraveno a převzato z \cite{OpenStax})}
		\label{img:hr_regulation}
	\end{center}
\end{figure}

Srdeční odezva na tyto nervové podněty je realizována na nejnižší intrakardiální
úrovní pomocí srdečních ganglií, které se skládají z neuronů. Přesněji jsou to
hlavně cholingerní (vagální) a adrenergní (sympatické) srdeční neurony, sloužící
jako vagální spínací body se schopností reakce na chemické, mechanické a
elektrofyziologické podněty. Díky tomu může například nervová aktivita v
prefrontální kůře modulovat HRV (\ref{section:hrv}). Blíže tyto vztahy popisuje
model Neuroviscerální integrace (NVI) \cite{Smith2017}.

Vliv parasympatiku je realizován uvolňováním mediátoru --- acetylcholinu --- z
koncových postgangliových vláken. Odpověď probíha v srdeční tkání díky
muskarinovým cholingerním recepotorům. Stimulací těchto receptorů se zpomaluje
proces spontánní diastolické depolarizace. To má za následek nižší srdeční
frekvenci (negativní chronotropie) v SA uzlu a zpomalení síňokomorového převodu
vzruchů v AV uzlu. Obecně tedy stimulace parasympatiku způsobuje včetně
zpomalení HR a síňokomorového převodu také snížení srdeční kontrakce a
excitability myokardu \cite{Kittnar2020}.

Stimulací sympatiku nastávají přesně opačné účinky než u parasympatiku. Ty
zahrnují zrychlení HR a síňokomorového převodu spolu se zvýšenou excitabilitou a
silou kontrakcí myokardu. Mediátorem je zde noradrenalin, který v
kardiomyocytech aktivuje \textbeta-adrenergní receptory, což vyvolává zrychlenou
spontánní depolarizaci. Výsledkem je primárně již zmíněná zvýšená srdeční
frekvence \cite{Kittnar2020}.

\paragraph*{\textit{Humorální regulace srdečního rytmu}\\} Regulace na této
úrovni vzniká vlivem hormonů, a to především působením katecholaminů ---
adrenalin a noradrenalin ---, produkovaných dření nadledvin nebo adrenergními
neurony sympatiku. Mají okamžitý nástup účinků, které jsou podobné těm jako u
stimulace sympatiku. Mezi další hormony ovlivňující srdeční frekvenci patří také
například hormony štítné žlázy, tyroxin a trijodthyronin
\cite{Kittnar2020,Orel2019}.

\paragraph*{\textit{Další faktory regulující srdeční rytmus}\\} Dalšími vlivy,
které ovlivňují srdeční frekvenci jsou například: koncentrace různých
elektrolytů v těle, tělesná teplota, rovnováha pH, dýchání, fyzická zátěž nebo
různé druhy kognitivní zátěže. Dále změny krevního tlaku, na které jsou citlivé
baroreceptory umístěné v oblouku aorty (arcus aortae) neboli tzv. reflexní
řízení. Mimo jiné se tato frekvence liší i věkem, pohlavím či zdravotními
podmínkami \cite{Kittnar2020}.

\subsection{Elektrokardiografie}
\label{section:electrocardiography}
Jednou z nejčastějších neinvazivních diagnostických metod v klinické praxi,
hlavně v oblasti kardiologie, je záznam a interpretace elektrické aktivity srdce
neboli elektrokardiografie. Princip této metody spočívá v měření elektrických
projevů srdeční aktivity na povrchu lidského těla. Každá perioda srdečního cyklu
je na buněčné úrovní doprovázená genezí nerovnoměrného elektrického napětí (AP),
což má za následek vznik místních elektrických proudů, resp. elektrického pole v
okolí myokardu. Vzniklé elektrické pole je zde ve skutečnosti součtem
jednotlivých elektrických polí každé srdeční buňky, kterou je možno vyjádřit
elementárním vektorem. Elementární vektory vyjadřují velikost a směr
elektrického pole. Sumací vektorů v jednom časovém momentu vzniká okamžitý
vektor elektrického pole jehož orientace a velikost charakterizují výslednou
naměřenou amplitudu v specifickém svodu, viditelnou na EKG křivce
\cite{Surawicz2008,Stejfa2006,Kittnar2020}.

Tkáně v lidském těle díky jejich elektrickým vlastnostem plní při styku s
elektrickým polem úlohu vodiče a díky tomu je možné mezi různými místy na
povrchu těla naměřit napětí. K snímaní napětí na rozhraní kůže se používají
elektrody. Zaznamenáváním výsledné hodnoty naměřených rozdílů potenciálu mezi
elektrodami v čase vzniká tzv. elektrokardiogram. Měření se liší počtem
použitých svodů a jejich lokalizací na lidském těle. Elektrokardiografii je tedy
možné kategorizovat z hlediska počtu, zapojení a umístění svodů do těchto
skupin: \cite{Haberl2012,Stejfa2006,Kittnar2020}:

\begin{enumerate}
	\item \textbf{Einthovenovy bipolární končetinové svody (I, II, III)} ---
	       vytváří tzv. teoretický Einthovenovův rovnoramenný trojúhelník, jehož
	       přibližným středem je srdce. Elektrody se v tomto případě většinou
	       nachází na horních končetinách a levé dolní končetině, přičemž dvě z
	       nich jsou elektrody aktivní. Měřenou amplitudu udává rozdíl
	       potenciálu aktivních elektrod a zároveň zde platí Einthovenovův
	       zákon, který říká že vektorový součet všech amplitud je roven nule.
	       Tyto svody jsou často nazývané standardními \cite{Kittnar2020}.
	      \begin{figure}[h]
		      \begin{center}
			      \includegraphics[width=0.8\textwidth]{../assets/anatomy/bipolar}
			      \caption{Bipolární končetinové svody \cite{Kittnar2020}}
			      \label{img:bipolar}
		      \end{center}
	      \end{figure}
	\item \textbf{Goldbergerovy unipolární končetinové svody (aVR, aVL, aVF)}
		  --- vznikají spojením aktivní elektrody s Wilsonovou svorkou. K
		  Wilsonově neobli nulové svorce jsou připojeny všechny končetinové
		  svody přes odpor ke zajištění nulového potenciálu na této svorce.
		  Pozdějí byl tento způsob upraven Goldbergerem, který touto modifikací
		  zesílil amplitudu svodů na záznamu. Zesílení vzniká odpojením aktivní
		  elektrody od Wilsonovy svorky čimž je následně měřen potenciál pouze
		  mezi odpojenou elektrodou a těmi zapojenými. Na Wilsonově svorce tímto
		  vzniká potenciál a dochází ke zvýšení amplitudy \cite{Kittnar2020}.
		\begin{figure}[h]
			\begin{center}
				\includegraphics[width=0.8\textwidth]{../assets/anatomy/unipolar1}
				\caption{Unipolární končetinové svody \cite{Kittnar2020}}
				\label{img:unipolar1}
			\end{center}
		\end{figure}
	\item \textbf{Wilsonovy unipolární hrudní svody (V1 -- V6)} --- sestávají z
		šesti hrudních elektrod zapojených proti referenční Wilsonově svorce.
		Wilsonova nulová svorka je zde opět tvořená spojením končetinových svodů
		přes odpor. Změnou v tomto zapojení je přechod z frontální roviny měření
		elektrické aktivity srdce do horizontální. V kombinaci s předešlími
		Goldbergerovy svody tak vzniká prostorová informace o elektrickém
		srdečním poli \cite{Kittnar2020}.
		\begin{figure}[h]
			\begin{center}
				\includegraphics[width=0.35\textwidth]{../assets/anatomy/unipolar2}
				\caption{Unipolární hrudní svod \cite{Kittnar2020}}
				\label{img:unipolar2}
			\end{center}
		\end{figure}
\end{enumerate}

Kombinací všech výše zmíněných svodů vzniká standardní, běžně používaný,
12-svodový EKG záznam. Graficky se zaznamenává na milimetrový papír nebo je
vidět na obrazovce v digitalizované podobě.


\subsection{Zpracování záznamu srdeční aktivity}
\subsubsection{Nežádoucí elementy a artefakty EKG}
\subsubsection{Metody zpracování}
\subsubsection{Rozdíly mezi offline a online zpracováním signálu}

\subsection{Variabilita srdeční frekvence}
\label{section:hrv}

\subsubsection{Metody hodnocení HRV}
\subsubsection{Korelace kvantitativních parametrů HRV s fyziologickými veličinami}
\subsubsection{Model neuroviscerální integrace}
Tento neustále doplňující se model, popisuje vztahy v rámci periferní
psychofyziologie, neurověd a autonomních funkci spjatých s regulací vagálního
tonu. Tyto znalosti a vztahy jsou kombinovány do několika detailně popsaných
teoretických vrstev, dohromady tvořících jediný framework sloužící také jako
prediktivní model, který pak usnadňuje vyhodnotit souvislosti a důsledky
zásadních fyziologických změn srdeční aktivity či veličin s ní spojených.

Tyto souvislosti mají například za následek, že odlišné druhy kognitivní zátěže
se promítají do frekvenčních domén. Specificky se projevují jako velmi nízko,
nízko a vysoko frekvenční komponenty (VLF, LF, HF), které spolu s jejich
vzájemnými poměry slouží pří samotné analýze jako indikátory činnosti
vegetativní nervové soustavy (VNS). Aby bylo ale možné tyto komponenty získat a
porovnávat či jakkoliv hodnotit srdeční aktivitu, musí se záznam srdeční
aktivity patřičně zpracovat.
\subsubsection{Praktické aplikace HRV v diagnostice a terapii}

Záznam srdeční aktivity neboli elektrokardiogram (EKG), na kterém vidíme časový
průběh změn elektrického potenciálu srdce, hraje velmi důležitou roli v
kardiologické diagnostice. Uchovává v sobě komponenty jak v časové, tak i ve
frekvenční doméně, díky kterým jsme schopni provádět EKG analýzu, a to i za
jinými účely než jen z hlediska kardiologie. Abychom ale mohli tuto analýzu
provést či extrahovat potřebné komponenty, musíme signál patřičně zpracovat,
jelikož jeho surová podoba může být zkreslená a často obsahuje mnohočetné
nežádoucí artefakty.

Důležitými východisky, od kterých se odvíjí následně zvolená metodika při
zpracování signálu a tím pádem i samotný výstup, jsou zejména kvalita a
charakteristika biosignálu. V ideálním případě se nabízí myšlenka univerzálního
způsobu, který signál spolehlivě zbaví všech nežádoucích elementů nehledě na
zmíněné východiska. Jelikož v posledních letech nastal velký průlom v oblasti
využití strojového učení (machine learning) a umělé inteligence (AI, artificial
inteligence) pro zpracování biosignálů, tak úvaha v rámci metod z těchto
oblastí může právě k takové zavádějící myšlence vést. V praxi se ale při
zpracování EKG signálu neorientuje jen jeho kvalitou či charakterem ale také
povahou nadcházející analýzy. Proto se při zpracování využívají variace,
kombinace a obdoby konvenčních metod, kde každou volíme na základě zvolené dílčí
oblasti analýzy nebo jiných specifických požadavků. Průnik všech metod obsahuje
v první řadě snahu o efektivní potlačení nežádoucích prvků a odkrývá tak část
problematiky této práce.

Při samotném základním EKG vyšetření srdce, zde hraje roli několik vnějších i
vnitřních vlivů, které se ve výsledku mohou jevit jako stěžejní při zpracovávání
biosignálu. Takové jevy můžeme právě nazvat nežádoucími prvky. Těmi jsou zejména
elektrické a magnetické vlastnosti tkání, zvláště svalový akční potenciál (AP),
nebo umístění a vodivost elektrod využívaných při vyšetření. Proto je velmi
důležité signál pečlivě analyzovat a filtrovat exaktními metodami, jinak by jeho
použití mohlo vést k vágním výsledkům. 

Po správném zpracování EKG signálu je možné začít s jeho analýzou. Její
interpretace umožňuje detekovat potencionální srdeční vady či jiné srdeční
stavy. Dále také můžeme pracovat s extrahovanými komponenty, s jejichž pomocí
lze například určit variabilitu srdečního rytmu (HRV, Heart rate variability).
Tato specifická metrika a analýza jejích parametrů má pro nás mnoho dalších
klinických významů a umožňuje využití nejen v rámci kardiologie ale také
neurologie a psychofyziologie.

V této práci se zaměříme na problematiku týkající se zpracování a analýzy EKG
signálu, a to nejen v rámci naměřených signálů (offline) ale také při měření v
reálném čase (online). Vybranou dílčí oblastí analýzy je především detekce a
hodnocení kognitivní zátěže z HRV společně se statistickými parametry,
vypočtenými v časové oblasti, které jsou založené na intervalech beat-to-beat
(N-N).

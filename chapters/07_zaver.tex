V rámci bakalářské práce bylo navrženo a realizováno softwarové MATLAB
řešení pro offline zpracování a hodnocení srdeční aktivity. Řešení je založeno
na postupu, který se skládá z částí předzpracování, detekce komponentů,
zpracování komponentů a analýzy. Za účelem detekce komponentů byl zaveden a
modifikován QRS detektor využívající adaptivní prahování. Pro zpracování
komponentů byl implementován komplexní algoritmus, který detekuje a koriguje
artefakty v rámci detekovaných komponentů. Hodnocení srdeční aktivity probíhá v
časové oblasti a je založeno na nelineární geometrické metodě Poincarého grafu.
Předmětem analýzy je variabilita srdeční frekvence. Realizované řešení
jednotlivé částí vizualizuje společně s Poincarého grafem a automaticky počítá
jeho kvantitativní parametry využitím metody proložení elipsou. 

Dále byla navržena a naprogramována multiplatformní aplikace v prostředí Python
pro online hodnocení srdeční aktivity. Pro aplikaci bylo vytvořeno grafické
uživatelské rozhraní, které uživateli umožňuje jednoduché interaktivní ovládání
aplikace. Software je založený na zpracování EKG signálu v reálném čase, který
je přijímán z měřicího zařízení přes bezdrátovou lokální síť. Zároveň v reálném
čase probíhá detekce komponentů, ze kterých jsou vypočteny vybrané parametry 
z časové oblasti. Aplikace poskytuje živou vizualizaci určených parametrů a 
zpracovaného EKG signálu na jejím hlavním panelu. Byla přidána i funkcionalita 
pro sběr surových nebo zpracovaných dat.

Pomocí navržené Python GUI aplikace bylo provedeno pilotní měření EKG na 
kontrolní skupině 5 probandů. Použitím realizovaného MATLAB řešení
byly záznamy zpracovány a analyzovány. Jednalo se o krátkodobé 10 minutové záznamy.
Na základě výsledků analýzy byly určeny sledované veličiny a vyšetřeny rozdíly mezi
úseky EKG záznamů, kdy byl proband v klidu nebo vystaven kognitivní zátěží. 

Bylo realizováno ověření navržených řešení statistickým zpracováním vyšetřených
rozdílů sledovaných veličin. V rámci celé kontrolní skupiny byl pozorován
statisticky významný rozdíl sledované veličiny SD1, do které se promítají
krátkodobé změny HRV podmíněné vlivem parasympatiku. Dominance parasympatiku je
spojena s aktivitou prefrontálního kortexu, který lze stimulovat kognitivní
zátěží. Ke stimulaci kognitivní zátěže byl u kontrolní skupiny využit Stroopův
test. Zjištěné výsledky potvrzují spolehlivost analýzy realizovaného řešení a
možnost využití Poicarého grafu jako kvantitativního ukazatele změn v ANS. 

\subsection{Budoucí práce}
V rámci projektu \textit{Virtuální město} (viz sekce~\ref{section:online_processing}) 
bude aplikce \textit{BBPM} nadále vyvíjena. Pro zvýšeni uživatelské přívětivosti budou 
některé funkce aplikace automatizovány. Například automatické vyhledávání a připojení 
měřícího zařízení na lokální bezdrátové sítí. Aplikace bude rovněž rozšířena o lokální
měření přes sériové porty, aby bylo možné používat k měření i zařízení, která nejsou
bezdrátové. Předpokladem je i rozšíření sledovaných veličin během hodnocení v reálném 
čase a potenciální implementace offline analýzy záznamů v časové nebo frekvenční oblasti. 

Vzhledem k čerstvému vydání nové verze frameworku \textit{PySide6}, který je využíván v rámci 
aplikace \textit{BBPM} (viz sekce~\ref{section:pyside}), dojde pravděpodobně k přechodu na 
tuto novější verzi. Na základě tohoto přechodu bude pozměněn i životní cyklus aplikace.

Dále se plánuje rozšíření dosavadně používaných formátů pro ukládání dat (CSV) o formát 
DICOM (Digital Imaging and Communications in Medicine). Formát DICOM je standardní 
formát pro zobrazování, distribuci a uchovávání medicínských dat. 


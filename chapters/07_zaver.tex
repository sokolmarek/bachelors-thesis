V rámci bakalářské práce bylo navrženo a realizováno softwarové MATLAB
řešení pro offline zpracování a hodnocení srdeční aktivity. Řešení je založeno
na postupu, který se skládá z částí předzpracování, detekce komponentů,
zpracování komponentů a analýzy. Za účelem detekce komponentů byl zaveden a
modifikován QRS detektor využívající adaptivní prahování. Pro zpracování
komponentů byl implementován komplexní algoritmus, který detekuje a koriguje
artefakty v rámci detekovaných komponentů. Hodnocení srdeční aktivity probíhá v
časové oblasti a je založeno na nelineární geometrické metodě Poincarého grafu.
Předmětem analýzy je variabilita srdeční frekvence. Realizované řešení
jednotlivé částí vizualizuje společně s Poincarého grafem a automaticky počítá
jeho kvantitativní parametry využitím metody proložení elipsou. 

Dále byla navržena a naprogramována multiplatformní aplikace v prostředí Python
pro online hodnocení srdeční aktivity. Pro aplikaci bylo vytvořeno graficky
uživatelské rozhraní, které uživateli umožňuje jednoduché interaktivní ovládání
aplikace. Software je založený na zpracování EKG signálu v reálném čase, který
je přijímán z měřícího zařízení přes bezdrátovou lokální síť. Zároveň v reálném
čase probíhá detekce komponentů, ze kterých jsou vypočteny vybrané parametry.
Aplikace poskytuje živou vizualizaci určených parametrů a zpracovaného EKG
signálu na jejím hlavním panelu. Byla přidána i funkcionalita pro sběr surových
nebo zpracovaných dat.

Pomocí navržené Python GUI aplikace bylo provedeno pilotní měření surových EKG
záznamů na kontrolní skupině pěti probandů. Použitím realizovaného MATLAB řešení
byly záznamy zpracovány a analyzovány. Jednalo se o krátkodobé záznamy. Na
základě výsledků analýzy byly určeny sledované veličiny a vyšetřeny rozdíly mezi
úseky EKG záznamů, kdy byl proband v klidu nebo vystaven kognitivní zátěží. 

Bylo realizováno ověření navržených řešení statistickým zpracováním vyšetřených
rozdílů sledovaných veličin. V rámci celé kontrolní skupiny byl pozorován
statisticky významný rozdíl sledované veličiny SD1, do které se promítají
krátkodobé změny HRV podmíněné vlivem parasympatiku. Dominance parasympatiku je
spojena s aktivitou prefrontálního kortexu, který lze stimulovat kognitivní
zátěží. Ke stimulaci kognitivní zátěže byl u kontrolní skupiny využit Stroopův
test. Zjištěné výsledky potvrzují spolehlivost analýzy realizovaného řešení a
možnost využití Poicarého grafu jako kvantitativního ukazatele změn v ANS. 


Jedním z hlavních výsledků práce je realizace MATLAB SW řešení pro offline
hodnocení srdeční aktivity pomocí Poincarého grafu. Pro zajištění adaptivního
zpracování EKG záznamu a minimální chybovosti v rámci HRV analýzy byl vytvořen
algoritmus, který se skládá z následujících částí: předzpracování, detekce
komponentů, zpracování komponentů, vizualizace. Jednotlivě vizualizované části
lze vidět v Příloze B.

V částí předzpracování prochází EKG záznam digitální filtrací. Filtrace je
uskutečněna pomocí integrovaných funkcí prostředí MATLAB. Filtry se
navrhují automaticky v závislosti na charakteru signálu s konstantní hodnotou
útlumu nepropustného pásma 60~\si\dB. V každém případě je navržen filtr
nejmenšího řádu typu pásmová propust, jehož impulsní charakteristika se
zjednodušeně orientuje délkou signálu. Principiálně se počítá nejmenší řád FIR
filtru potřebný pro splnění jeho zadaných specifikací. FIR filtr je navržen a
uplatněn, pokud je vypočítaný řád alespoň dvakrát menší než délka signálu. V
případě nesplnění podmínky je stejný postup aplikován pro IIR verzi filtru, kde
musí být vypočítaný řád alespoň třikrát menší než délka signálu. Pokud není
vyhověno ani jedné podmínce je navržen IIR filtr s řádem odpovídajícím třetině
délky signálu. Fázové zpoždění filtrů se kompenzuje automaticky. Pro všechny EKG
signály v rámci této práce byl algoritmem automaticky navržen FIR filtr s
rozmezím dolního tolerančního pásma 6,32--7,50~\si\Hz, horního tolerančního
pásma 35,00--68,75~\si\Hz~a zvlněním 0,1~\si\dB~v propustném pásmu. Pro návrh
FIR filtru je využívaná metoda Kaiserova okna (viz
sekce~\ref{section:online_processing}), která je mezi dalšími metodami oken pro
filtraci EKG signálu nejefektivnější. Metody byly již porovnané v
\cite{Kumar2014,Lakhwani2013,Yadav2011}. Adaptivní filtrace v podání
realizovaného řešení umožňuje ve srovnání s jinými open-source MATLAB
aplikacemi -- například~\cite{ecgkit}~a~\cite{Sedghamiz2018} -- širší a
spolehlivější možnosti filtrace EKG záznamů různorodého charakteru. Hlavním
důvodem je, že většina aplikací využívá pouze jeden specifický filtr s pevně
nastavenými parametry, které nelze měnit ani uživatelsky. Nejčastěji se lze
setkat s FIR filtry navržené metodou nejmenších čtverců nebo metodou okna a IIR
Butterworthovy filtry typu pásmová propust. V těchto případech tak vzniká
předpoklad, který vymezuje použití aplikace pro určitou kategorii EKG signálů.
Teoretickým příkladem může být Pan-Tompkinsův algoritmus (viz
sekce~\ref{section:components_detection_theory}), který u filtrace využívá nízké
mezní frekvence pásmové propusti (5--15~\si\Hz) a primárně předpokládá EKG s
normálním sinusovým rytmem. Při použití EKG záznamů s výskytem arytmií, které se
zpravidla filtrují s vyšší horní mezní frekvencí, je efektivita algoritmu
nižší~\cite{Fariha2020}. Stejná situace nastává také v případě záznamů, které
jsou velmi zatížené svalovými artefakty (viz
kapitola~\ref{section:artifacts_theory}). Vzhledem ke komplexnosti řešení je
toto snížení v případě Pan-Tompkinsovy metody oproti jednodušším metodám poměrně
zanedbatelné a lehce kompenzovatelné. U jednodušších metod by však nebylo možné
takto zpracovaný záznam použit k analýze. Proto pro zajištění vysoké
spolehlivosti analýzy EKG záznamů je několik kroků předzpracování a povaha
detekce komponentů navrženého řešení inspirovaná Pan-Tompkinsovou metodou.
Výsledek jednotlivých kroků zpracování EKG lze vidět na
Obr.~\ref{fig:preprocessing_steps}. Zároveň je v navržené aplikaci uživateli
umožněno měnit mezní frekvence pásmové propusti, což společně s adaptivní
metodou návrhu filtrů rozšiřuje využití realizovaného řešení.

Detekce komponentů využívá podobně jako Pan-Tompkinsův QRS detektor dva
adaptivní prahy. První práh vychází z amplitud R vln a druhý z délky R-R
intervalů. Detekční algoritmus (viz kapitola~\ref{section:components_detection})
není v tomto případě příliš sofistikovaný a počítá především s kvalitně
zpracovaným a vyhlazeným EKG záznamem. To je v rámci realizovaného řešení
zajištěno rozšířením předzpracování o zpětnou kumulaci umocněného signálu a
následné vyhlazení integračním filtrem. Výsledkem je signál v podobě špičatých
vrcholů s kladnou amplitudou, které lze vizuálně přirovnat ke Gaussově funkci s
velmi nízkou hodnotou parametru~\textsigma. Pro porovnání a ověření
spolehlivosti detekce byl implementován Pan-Tompkinsův QRS
detektor~\cite{Pan1985} a jeho modifikovaná Hamiltonova
verze~\cite{Hamilton1987}. Ve všech případech použití byl výsledek detekce u EKG
záznamů sledovaných osob stejný. Je třeba podotknout že pilotní měření záznamu
bylo uskutečněno za laboratorních podmínek a tomu odpovídala i relativně dobrá
kvalita signálů. Detekce by tedy mohla být pochybná v případě EKG záznamů
zatížených mnohočetnými artefakty nebo projevy kardiovaskulárních onemocnění.
Vzniká tak zde prostor na zlepšení v komplexnosti a rozsahu použití detekčního
algoritmu pro případné budoucí využití. Zásadní problém použitého detekčního
algoritmu může nastat například v situaci abnormální amplitudy R vlny, která
bude z časového hlediska normální. Jelikož se první práh adaptuje průměrováním
předešlých amplitud R vln, abnormálně vysoká či nízká amplituda by mohla práh
posunout natolik, že by detekce následujících normálních R vln byla zanedbána.
Řešením může být identifikace a eliminace odlehlých hodnot (outliers)
amplitud v cyklické frontě, ze které se průměrem počítá první práh nebo zvolení
nekauzálního přístupu prahování v závislosti na offline charakteru zpracování.

Výsledkem části detekce komponentů je časová série R vln, která je vzápětí
zpracovaná a korigovaná. Korigované výsledky jsou uvedeny v
Tab.~\ref{tab:corrected_components}. Detekované artefakty každého záznamu lze
vidět v Tab.~\ref{tab:detected_artifacts}. Pro zpracování detekovaných
komponentů byla použita metoda, navržená firmou Kubios, která se specializuje na
zpracování a analýzu variability srdečního rytmu. Jedná se o sofistikovaný
algoritmus (viz sekce~\ref{section:components_processing}), který byl
implementován za účelem maximalizace spolehlivosti hodnocení srdeční aktivity
pomocí Poincarého grafu. V současnosti jsou softwarové řešení firmy Kubios
pravděpodobně nejpopulárnější a nejrozšířenější ve vědeckém výzkumu, a i v
klinické praxi. Kubios poskytuje včetně komerčních softwarových řešení i
aplikaci jménem \textit{Kubios HRV Standard}, kterou lze využít zdarma pro
osobní použití. Mezi funkcemi aplikace je i nelineární HRV analýza Poincarého
grafem. Srovnáním výstupních reportů analýzy aplikace \textit{Kubios HRV
    Standard} a výsledků podle Tab.~\ref{tab:corrected_components} byla ověřena
účinnost a správná funkcionalita implementovaného algoritmu spolu s vypočítanými
indexy Poincarého grafu. Jednotlivé reporty jsou součástí obsahu přiloženého
datového nosiče. Komplexní zpracování detekovaných komponentů zároveň do jisté
míry kompenzuje potenciální chyby zapříčiněné detekčním algoritmem.

Jiné, volně dostupné MATLAB aplikace, určené pro analýzu HRV --
například~\cite{Ramshur2010} nebo~\cite{Kardia2010} -- poskytují rozmanitější
možnosti analýzy, a to jak ve frekvenční, tak časové doméně. Jejich
předpokládaným vstupem je ale už zpracovaná, ideálně i korigovaná, časová řada
R-R intervalů. Tyto řešení samy o sobě nejsou schopné zpracovat EKG záznam a
nevyužívají ani velmi komplexních metod pro korekci artefaktů R-R intervalů.
Zpracování v rámci těchto aplikací se obvykle týká především odstranění trendu
vstupních dat k zajištění stacionarity signálu pro potřeby spektrální a
fluktuační HRV analýzy. Rozšířením možností analýzy realizovaného řešení by tedy
bylo možné získat nadhled nad mnoha open-source MATLAB aplikacemi či skripty. To
vyplývá i z faktu, že většina současných řešení je zastaralá a nové nepřibývají.
Jedním z důvodu může být prudce vzrůstající popularita prostředí Python, které v
mnoha případech čím dál častěji nahrazuje MATLAB realizace.

Dalším výsledkem práce je naprogramované SW řešení s GUI pro online hodnocení
srdeční aktivity v prostředí Python. Aplikaci \textit{BBPM} lze vidět
na Obr.~\ref{fig:results_bbpm}. Popis programu je uveden v
kapitole~\ref{section:online_data_process}. Filtrace signálu a detekce
komponentů (viz sekce~\ref{section:online_processing}) zde nejsou zdaleka tak
propracované jako v předešlém MATLAB řešení. Vzhledem k budoucímu
používaní aplikace je tedy žádoucí, minimálně v těchto dvou ohledech, aplikaci
vylepšit. Během zpracovávání dat v reálném čase se výsledek filtrace SGF jeví po
vizuální stránce uspokojivě, avšak zvolená metoda filtrace může vést v mnoha
případech k závažné chybě. Filtr SGF využívá centrované plovoucí okno, a počítá
tak s okolními hodnotami. Délka okna je staticky nastavený parametr v závislosti
na datech přijatých z použitého měřícího zařízení (viz
kapitola~\ref{section:measurement_device}). Aplikace ze zařízení přijímá datové
bloky o velikosti 2048 vzorků. Filtrace by nyní nefungovala při použití aplikace
s jiným zařízením, které posílá hodnoty po jednom vzorku nebo po blocích menších
než nastavená délka okna. Nabízí se tak implementace variabilnějšího řešení v
podobě rekurzivní IIR filtrace. Filtry IIR jsou běžně využívané k filtraci
signálu v reálném čase už jen díky jejich nízké výpočetní náročností. Pro
srovnání s FIR filtrací využívanou v MATLAB řešení byl navržen
Eliptický IIR filtr se stejnými parametry, který byl ve filtraci každého
naměřeného signálu průměrně~8,5x rychlejší. Detekce komponentů je realizovaná
pouze hledáním lokálních maxim signálu, které mají charakteristiky R vln --
amplitudu a prominenci. Aplikaci \textit{BBPM} tak v budoucím vývoji
pravděpodobně nemine zavedení robustního QRS detektoru. Tím může být některá
modifikace (již zmíněného) Pan-Tompkinsonova QRS detektoru, který byl primárně
navržen pro uplatnění v reálném čase. Pro účely této práce byla aplikace
\textit{BBPM} použita pro sběr surových dat. Naměřená data jsou součástí obsahu
přiloženého datového nosiče.

Pro ověření naprogramovaných řešení bylo realizováno měření EKG záznamů na
kontrolní skupině 5 osob (viz sekce~\ref{section:probands}). Měření probíhalo
podle postupu uvedeného v sekci~\ref{section:measurement_methodology}. Záznamy
byly naměřeny aplikací \textit{BBPM}. Každý EKG záznam byl zpracován
realizovaným MATLAB řešením a vyhodnocen použitím nelineární
geometrické metody, Poincarého grafu (viz kapitola~\ref{section:hrv_methods}). V
rámci všech probandů byly vypočítány průměrné hodnoty korigovaných R-R intervalů
a srdeční frekvence spolu s jejich směrodatnými odchylkami
(Tab.~\ref{tab:corrected_components}). Dále byly určeny indexy Poincarého grafu
SD1, SD2 a jejich poměr SD1/SD2 (Tab.~\ref{tab:poincare_parameters}). Všechny hodnoty byly
zároveň zvlášť vypočítány pro klidový úsek (N) EKG záznamu a pro úsek, kde byl
proband vystaven kognitivní zátěží (S) pomocí Stroopova testu (viz
sekce~\ref{section:stroop_test}). Dále proběhlo statistické zpracování výsledků
dle kapitoly \ref{section:statistical_methods}. Ve všech případech souborů
indexů Poincarého grafu byl splněn předpoklad normality dat
(Tab.~\ref{tab:normality_tests}). Následně bylo realizováno srovnání sledovaných
veličin mezi úseky N a S. Statisticky významný rozdíl byl pozorován (viz
Tab.~\ref{tab:t_tests}) pouze u veličin SD1 ($p=$0,0132). Grafické srovnání
výsledků sledovaných veličin lze vidět na Obr.~\ref{fig:results_sd_vals}.

Stroopův test (sekce~\ref{section:stroop_test}) se v klinických studiích obvykle
používá ke stimulaci kognitivní zátěže, která se projevuje změnami v
parasympatickém (PNS) a sympatickém (SNS) nervovém systému (viz
kapitola~\ref{fig:hr_regulation})~\cite{Hoshikawa1997}. Důsledkem změn je
modulace HRV (kapitola~\ref{section:hrv}). Na základě
studií~\cite{Brennan2001,Kamen1996} bylo prokázáno, že lze změny ANS
kvantitativně hodnotit nelineární metodou Poincarého grafu. K hodnocení slouží
primárně indexy SD1 a SD2. Parametr SD1 reflektuje aktivitu PNS, tedy krátkodobé
změny HRV. Na parametru SD2 se projevují dlouhodobé změny HRV, tudíž souvisí s
aktivitou SNS. Poměr parametrů SD1/SD2 představuje sympatovagální
rovnováhu~\cite{Hsu2012,Habib2013,Mazhar2007}. V rámci této práce byly použity
krátkodobé záznamy EKG. Na výsledných datech analýzy Poincarého grafem
(Tab.~\ref{tab:poincare_parameters}) lze vidět pokles hodnot SD1 v úsecích
kognitivní zátěže, který byl statistický významný (Tab.~\ref{tab:t_tests}).
Výsledek je v souladu se
studiemi~\cite{Sebastiano2019,Brugnera2018,Vazan2017,Melillo2011}.





















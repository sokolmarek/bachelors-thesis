Bakalářská práce se věnuje návrhu a realizaci softwarového řešení pro hodnocení
srdeční aktivity v programovém prostředí MATLAB a Python. Hlavním cílem je
navrhnout adaptivní algoritmus v prostředí MATLAB, který bude schopný realizovat
offline analýzu na naměřeném EKG signálu zatíženého artefakty. Pro analýzu
zpracovaného záznamu byla vybrána nelineární geometrická metoda Poincarého
grafu, kterou je hodnocena variabilita srdeční frekvence. Dalším cílem práce je
navrhnout a realizovat řešení s GUI pro online hodnocení srdeční aktivity v
prostředí Python. K testování navrženého řešení byly použity krátkodobé záznamy
naměřené celkem u pěti probandů. Během měření byli probandi nejdříve v klidu a
následně byl každý z nich vystaven situaci stimulující kognitivní zátěž. Každý z
probandů byl během měření monitorován Holterem neboli přenosným
elektrokardiografem. Výsledkem práce je MATLAB aplikace schopná offline
adaptivně zpracovat EKG záznam a zobrazit grafy parametrů prokazujících korelaci
kognitivní zátěže s variabilitou srdečního rytmu v závislosti na čase. Byla
naprogramována multiplatformní Python GUI aplikace rozšiřující výstup v rámci
online měření a zpracování EKG záznamů. 


Bakalářská práce se věnuje návrhu a realizaci softwarového řešení pro hodnocení
srdeční aktivity v programovém prostředí MATLAB a Python. Hlavním cílem je
navrhnout adaptivní algoritmus v prostředí MATLAB, který bude schopný realizovat
offline analýzu na naměřeném EKG signálu zatíženého artefakty pomocí Poincarého
grafu. Pro hodnocení zpracovaného záznamu byla vybraná analýza v časové oblasti,
konkrétně variabilita srdeční frekvence a z ní vycházející parametry. Dalším
cílem práce je navrhnout a realizovat řešení s GUI pro online hodnocení srdeční
aktivity v prostředí Python. K testování navrženého řešení byly použity
krátkodobé záznamy naměřené celkem u 5 probandů. Během měření byli probandi
nejdříve v klidu a následně byl každý z nich vystaven situaci stimulující
kognitivní zátěž. Každý z probandů byl během měření monitorován Holterem neboli
přenosným elektrokardiografem. Výsledkem práce je sada skriptů implementovaných
v prostředí MATLAB schopných offline adaptivně zpracovat EKG a zobrazit grafy
parametrů prokazujících korelaci kognitivní zátěže s variabilitou srdečního
rytmu v závislosti na čase. Byla naprogramována multiplatformní Python GUI
aplikace rozšiřující výstup v rámci online měření. 


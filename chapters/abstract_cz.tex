Bakalářská práce se věnuje návrhu a realizaci softwarového řešení pro hodnocení
srdeční aktivity v programovém prostředí MATLAB a Python. Hlavním cílem je
navrhnout adaptivní algoritmus, který bude schopný realizovat analýzu na
naměřeném EKG signálu zatíženého artefakty a dále aplikovat tuto metodu pro
měření v reálném čase. Pro hodnocení zpracovaného záznamu byla vybraná analýza v
časové oblasti, konkrétně variabilita srdeční frekvence a z ní vycházející
parametry. Dalším cílem práce je vizualizace výstupu v podobě interaktivních
grafů zobrazujících ve vybraných časových úsecích Poincarého graf. K testování
navrženého řešení byly použity krátkodobé záznamy naměřené celkem u 5 probandů.
Během měření byli probandi v prostředí virtuální reality, ve které byl každý z
nich vystaven situaci stimulující kognitivní zátěž a monitorován přenosným
elektrokardiografem neboli Holterovsky. Výsledkem práce je sada skriptů
implementovaných v prostředí Matlab schopných offline adaptivně zpracovat EKG a
zobrazit grafy parametrů prokazujících korelaci kognitivní zátěže s variabilitou
srdečního rytmu v závislosti na čase. Byla naprogramována multiplatformní Python
GUI aplikace rozšiřující výstup v rámci online měření.
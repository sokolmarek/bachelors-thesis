The bachelor's thesis deals with the design and implementation of a software
solution for the evaluation of cardiac activity in the MATLAB and Python
software environment. The main goal is to design an adaptive algorithm that will
be able to perform analysis on the measured ECG signal loaded with artifacts and
further apply this method for real-time measurement. For the evaluation of the
processed records, the analysis in the time domain was selected, specifically
the variability of the heart rate and the parameters based on it. Another goal
of the work is the visualization of the output in the form of interactive graphs
showing a graph of Poincaré in selected time periods. For testing of the
proposed solution, short-term records measured in a total of 5 probands were
used. During the measurements, the probands were in a virtual reality
environment in which each of them was exposed to a situation that stimulates
cognitive stress and was monitored by Holter, a portable electrocardiogram. The
result of the work is a set of scripts implemented in the MATLAB environment
capable to adaptively process the ECG offline and display graphs of the
parameters demonstrating the correlation of cognitive load with heart rate
variability over time. A multiplatform Python GUI application was programmed to
extend the output for online measurement.
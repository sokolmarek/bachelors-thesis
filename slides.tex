% % % % % % % % % % % % % % % % % % % % % % % % % % % % % % 
%
%       Šablona pro prezentaci k SZZ, FBMI ČVUT
% 		by Marek Sokol :)))		
%  
% % % % % % % % % % % % % % % % % % % % % % % % % % % % % % 

% arara: xelatex: { shell : yes }

\documentclass[czech,aspectratio=169,12pt]{beamer}   % nastavení formátu prezentace 16:9 (beamer)

% Nastavení vzhledu prezentace
% Hezká dokumentace od overleafu: https://www.overleaf.com/learn/latex/beamer
\usetheme{Madrid}
\usecolortheme{whale}
\useinnertheme{rectangles}                      % možnosti: default,circles,rectangles,rounded,inmargin
\useoutertheme{default}                         % možnosti: default,miniframes,smoothbars,sidebar,split,shadow,tree,smoothtree,infolines
\definecolor{CVUT}{HTML}{0065BD}                % definice ČVUT modré barvy
\setbeamercolor{structure}{bg=white,fg=CVUT}    % nastavení barvy

% Zapnutí navigačního panelu prezentace (pro vypnutí odkomentujte)
%\beamertemplatenavigationsymbolsempty

% Generování slidu s poznámkami:
% Pro vygenerovaní poznámek odkomentujte příkaz níže
%\setbeameroption{show notes}

% Základní balíčky
\usepackage[english,main=czech]{babel}          % čeština
\usepackage{fontspec}                           % specifické fonty
\setmainfont{Arial}
\usepackage{graphicx}
\usepackage{hyperref}                           % odkazy
\usepackage{tikz}                               % pro kreslení a vizualizaci tvarů and shit
\usetikzlibrary{chains,fit,shapes}

% Deklarace názvů - přepsat dle autora
\title[Zpracování a analýza záznamu srdeční aktivity]{Zpracování a analýza záznamu srdeční aktivity}
\subtitle{Bakalářská práce}
\institute[FBMI ČVUT v~Praze]{Fakulta biomedicínského inženýrství \\ České vysoké učení technické v~Praze}
\author[M. Sokol]{Marek Sokol \\ Vedoucí práce: Mgr. Ksenia Sedova, Ph.D.}
\date{1. 1. 2020}
\titlegraphic{\includegraphics[width=.1\textwidth]{assets/slides/logo-cvut}}

% Příklad slidu s poznámkou
\begin{document}
\begin{frame}
    \titlepage
    \note{Nezapomenout pozdravit}
\end{frame}

% Generování slidu s obsahem - třeba použít pak u obsahu slidu \section, \subsection jinak nebude nic vidět :)))
% \begin{frame}{Obsah}
%     \tableofcontents
% \end{frame}

\begin{frame}{Test frame bez odrážek}
    \begin{center}
        50 \% lidí má tendenci přejít k temné straně
        \vskip5mm
        Barva lightsaberu neodpovídá barvě očí\\
        Impérium nebo rebelové?\\
    \end{center}
\end{frame}

\begin{frame}{Test frame s odrážkami}
    \begin{itemize}
        \item mango1
        \item \textbf{mango2}
        \item mango3
    \end{itemize}
\end{frame}

\begin{frame}{Test frame s obrázkem}
    \begin{center}
        \includegraphics[width=.6\textwidth]{assets/slides/logo-cvut}
    \end{center}
\end{frame}

\begin{frame}{Test frame s citací}
    \begin{center}
        {\large ``I find your lack of faith disturbing.''}
        \vskip5mm
        --- Darth Vader
    \end{center}
\end{frame}

\begin{frame}{Test frame se shrnutím}
    \begin{itemize}
        \item Užasna prezentace
        \item Imperium padlo
        \item Mandalorian skončil
        \item Bude serial Kenobi!
        \item Disney trilogie je příšerná
        \item Lucas se plácá po čele
    \end{itemize}
\end{frame}

\begin{frame}[noframenumbering]{Otázky oponenta}
    Otázka první:
    Jakým jazykem mluví Ewokové?

    \vfill

    Odpověď: Mluví kombinací tibetštiny a nepálštiny.
\end{frame}

\end{document}
